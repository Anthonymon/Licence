\documentclass[a4]{article}
\usepackage[utf8]{inputenc}
\usepackage[french]{babel}
\usepackage{amssymb}
\title{Petit Memento de logique}

\begin{document}
\maketitle
La logique des propositions manipule des propositions. Ces propositions prennent des valeurs dans les 
booléens c'est-à-dire l'ensemble {VRAI,FAUX}.\\
On peut construire à partir de 2 énoncés, un énoncé plus compliqué grâce à des conneteurs logiques.\\

\begin{tabular}[h]{|c|}
\hline
Les connecteurs logiques \\
\hline
\end{tabular}\\


La négation($\urcorner$)

\vspace{0.1cm}
\begin{tabular}[h]{|c|c|}
\hline
$A$ & $\urcorner B$\\
\hline
VRAI & FAUX\\
\hline
FAUX & VRAI\\
\hline
\end{tabular}\\
\vspace {1cm}

La conjonction ($\wedge$)

\vspace{0.1cm}
\begin{tabular}[h]{|c|c|c|}
\hline
A & B & $A\wedge B$\\
\hline
VRAI & FAUX & FAUX\\
\hline
VRAI & VRAI & VRAI\\
\hline
FAUX & FAUX & FAUX\\
\hline
FAUX & VRAI & FAUX\\
\hline 
\end{tabular}\\
\vspace {1cm}

La disjonction($\vee$)

\vspace{0.1cm}
\begin{tabular}[h]{|c|c|c|}
\hline
A & B & $A\vee B$\\
\hline
VRAI & VRAi & VRAI\\
\hline
VRAI & FAUX & VRAI\\
\hline
FAUX & VRAI & VRAI\\
\hline
FAUX & FAUX & FAUX\\
\hline
\end{tabular}\\
\vspace{2cm}

L'implication($A \Rightarrow B$)

\vspace{0.1cm}
\begin{tabular}[h]{|c|c|c|}
\hline
A & B & $A\Rightarrow B$\\
\hline
VRAI & VRAI & VRAI\\
\hline
VRAI & FAUX & FAUX\\
\hline
FAUX & VRAI & VRAI\\
\hline
FAUX & FAUX & VRAI\\
\hline
\end{tabular}\\
\vspace{1cm}

L'équivalence($A \Leftrightarrow B$)

\vspace{0.1cm}
\begin{tabular}[h]{|c|c|c|}
\hline
A & B & $A \Leftrightarrow B$\\
\hline
VRAI & VRAI & VRAI\\
\hline
VRAI & FAUX & FAUX\\
\hline
FAUX & VRAI & FAUX\\
\hline
FAUX & FAUX & VRAI\\
\hline
\end{tabular}\\
\vspace{1cm}

\begin{tabular}[h]{|c|}
\hline
Les lois de De Morgan \\
\hline
\end{tabular}\\


$\urcorner (A \wedge B) \Leftrightarrow \urcorner A \vee \urcorner B$

$\urcorner (A \vee B) \Leftrightarrow \urcorner A \wedge \urcorner B$\\
\vspace{1cm}

\begin{tabular}[h]{|c|}
\hline
Variables et quantificateurs \\
\hline
\end{tabular}\\


Dans une proposition mathématique, on peut utiliser une variable qui n'a pas de valeur
définie. Si la variable est $x$ on peut noté la proposition $A(x)$.
On peut lier les variables libres des propositions grâce à des quantificateurs. 
Il existe deux quantificateurs qui permettent d'utiliser des variables dans les expressions 
mathématiques : 
\begin {enumerate}
\item "Quel que soit" : $\forall x A(x)$ \\
\vspace {1.5cm}
signifie que la propriété $A(x)$ est vraie pour toutes les valeurs de $x$
\item "Il existe" : $\exists x A(x)$ \\
\vspace {1.5cm}
signifie qu'il existe au moins un $x$ qui satisfait la proposition $A(x)$\\
\end{enumerate}


\begin{tabular}[h]{l}

Equivalences :\\
\hline
\end{tabular}\\

\vspace{0.1cm}
$\urcorner (\forall x A(x)) \Leftrightarrow \exists x \urcorner A(x)$ 

$\urcorner (\exists x A(x)) \Leftrightarrow \forall x \urcorner A(x)$


\end{document}
