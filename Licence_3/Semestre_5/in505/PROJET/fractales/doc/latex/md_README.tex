C\+A\+U\+M\+ES Clément D\+O\+U\+D\+O\+UH Yassin Langages Avancés I\+N505

\section*{Remise du projet}


\begin{DoxyEnumerate}
\item Pour obtenir le listing commenté du programme, il suffit de taper la commande suivante \+:
\end{DoxyEnumerate}


\begin{DoxyCode}
$ make listing
\end{DoxyCode}



\begin{DoxyEnumerate}
\item Pour lire le compte rendu du projet, il faut taper la commande suivante \+:
\end{DoxyEnumerate}


\begin{DoxyCode}
$ make rapport
\end{DoxyCode}


\section*{Mécanisme d\textquotesingle{}installation et de manipulation de l\textquotesingle{}application}


\begin{DoxyEnumerate}
\item Pour vous assurez la compilation de l\textquotesingle{}application la première fois, il faut taper la commande suivante \+:
\end{DoxyEnumerate}


\begin{DoxyCode}
$ make first
\end{DoxyCode}



\begin{DoxyEnumerate}
\item Les commandes Makefile expliquées sont les suivantes \+:
\end{DoxyEnumerate}


\begin{DoxyCode}
$ make compil 
$ make run
$ make all
$ make clean 
$ make install 
\end{DoxyCode}


La première commande permet uniquement de compiler l\textquotesingle{}application \+: un exécutable fractale sera généré. La deuxième permet d\textquotesingle{}exécuter le projet. La troisième permet de compiler et d\textquotesingle{}exécuter successivement l\textquotesingle{}application. La quatrième commande permet de supprimer tous les fichiers liés à la compilation La dernière est utile uniquement à chaque changement de machine virtuelle.

\section*{Mode d\textquotesingle{}emploi}


\begin{DoxyEnumerate}
\item Choisir la fractale en cliquant sur le bouton Valider de l\textquotesingle{}onglet correspondant. Pour la fractale Julia et Fatou, le choix du nombre complexe associé est nécessaire.
\item La manipulation de la fractale peut se faire de plusieurs manières \+:
\end{DoxyEnumerate}
\begin{DoxyItemize}
\item la modification de la partie réelle et de la partie imaginaire du complexe représentant la fractale (pour le cas de Julia et Fatou).
\item le changement de la valeur de z\+Max.
\item l\textquotesingle{}enregistrement vectoriel avec Cairo en utilisant le bouton Enregistrer.
\item le bouton Quitter permet de fermer l\textquotesingle{}application.
\item le déplacement sur l\textquotesingle{}image avec les boutons Z/\+Q/\+S/D.
\item le changement de couleur de la fractale en appuyant successivement sur le bouton C
\item le zoom et le dezoom avec O et P
\item le changement de la granularité de l\textquotesingle{}image de la fractale avec L et M 
\end{DoxyItemize}