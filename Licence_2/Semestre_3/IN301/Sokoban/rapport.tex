% Clément CAUMES 21501810
% 09 janvier 2017
% PROJET SOKOBAN

% Fichier de rapport du projet SOKOBAN

\documentclass[a4]{article}

\usepackage[utf8]{inputenc}
\usepackage[french]{babel}

\author{Clément Caumes 21501810}
\title{Rapport du projet Sokoban}
\date{09 janvier 2017}

\begin{document}
\maketitle

\section{Description des structures de données}
\subsection{struct casePlateau ou CASE}
La structure \textbf{CASE} a deux champs qui correspondent aux différentes possibilités de ce que contient la case.
Deux champs étaient nécessaires car il fallait gérer la superposition d'éléments.
Le champ \textit{natureCase} peut contenir soit \emph{VIDE}, soit \emph{MUR}, soit \emph{EMPLACEMENT\textunderscore CAISSE}.
Le champ \textit{natureElem} peut contenir soit \emph{PERSO}, soit \emph{CAISSE}, soit \emph{PAS\textunderscore ELEM}.

\subsection{struct positionPerso ou POSITIONPERSO}
La structure \textbf{POSITIONPERSO} correspond à la position du personnage sur le plateau. 
Si le personnage n'est pas positionné (lors de l'interface de création) alors les champs \textit{x} et \textit{y} contiennent les valeurs \emph{INDEFINI}.

\subsection{struct plateau ou PLATEAU}
La structure \textbf{PLATEAU} possède un champ \textit{CASE** T} qui est un tableau dynamique du plateau à 2 dimensions, 
un champ \textit{POSITIONPERSO} pour stocker la position du personnage, un champ d'entiers nbCases qui correspond au nombre de cases sur une ligne du plateau.
Le nombre de cases réelles du plateau sera donc $nbCases \times nbCases$. Il y a un champ tailleCases qui est la taille des cases dans le plateau. 

\subsection{struct elem et HISTORIQUE}
La structure \textbf{HISTORIQUE} contient deux piles. Un élément d'une pile correspond à un \textit{PLATEAU} p, à
un entier nbCoups (égal au toucher de caisses) et à un pointeur vers l'élément suivant.

\subsection{struct actionCreation ou ACTIONCREATION}
La structure \textbf{ACTIONCREATION} correspond à l'état du sokoban lorsque l'utilisateur est en train de créer son propre niveau. 
Il possède un champ d'entiers mode qui peut être soit \emph{MODE\textunderscore EDITION} (lors de la création du niveau) soit \emph{MODE\textunderscore RESOLUTION} (lors du jeu à l'envers) soit \emph{MODE\textunderscore QUITTER}. 
Il y a aussi un tableau de caractères nomFichier qui contient le nom du fichier que l'utilisateur a tapé dans la ligne de commande. 
Enfin, il y a un \textit{PLATEAU} qui sera fixe lors de la création.

\subsection{struct actionJeu ou ACTIONJEU}
La structure \textbf{ACTIONJEU} correspond à l'état du sokoban lorsque l'utilisateur est en train de jouer à une partie d'un fichier.
Il possède un champ d'entier mode qui peut être \emph{MODE\textunderscore JEU} (quand on joue au Sokoban) ou \emph{MODE\textunderscore QUITTER}.
Le nomFichier stocke le nom du fichier sur lequel l'utilisateur est en train de jouer. 
Le num correspond au numéro du niveau en cours de jeu
Le niveauMax est le niveau maximal du fichier. 
\textit{HISTORIQUE} h permet de stocker l'historique du niveau en cours de jeu.Le nbCases stocke le nombre de cases en longueur (largeur du plateau de sokoban en cours de jeu).


\section{Description des algorithmes complexes}
\subsection{int teste\textunderscore perso\textunderscore interieur\textunderscore entrepot}
Ce test renvoie 1 si le personnage est entouré de murs, 0 sinon. Il est composé de 4 sous-fonctions : 
\begin{itemize}
\item initialise\textunderscore copie\textunderscore plateau\textunderscore test crée un plateau \textit{pCopie} de même taille que le plateau \textit{plat} en argument et met dans chaque case la valeur \emph{CASE\textunderscore NON\textunderscore TESTEE}.
\item met\textunderscore valeurs\textunderscore debut \textunderscore test met autour de la case en position du personnage de \textit{plat} les valeurs \emph{CASE\textunderscore TESTEE}.
\item diffuse\textunderscore valeurs\textunderscore test "transmet" au fur et à mesure la valeur \emph{CASE\textunderscore TESTEE} dans \textit{pCopie} aux cases voisines si elles ne sont pas des murs (de \textit{plat}).
\item test\textunderscore diffusion\textunderscore valeurs\textunderscore test regarde si les cases des bords du plateau de \textit{pCopie} sont \emph{CASE\textunderscore TESTEE}. Si oui, alors le personnage n'est pas enfermé.
\end{itemize}

\subsection{PLATEAU calcul\textunderscore plateau\textunderscore niveau}
Cette fonction renvoie un plateau qui correspond à la lecture du niveau num de \textbf{ACTIONJEU} a.
D'abord, on initialise un plateau de la taille nbCases en argument. Ensuite, il y a une recherche dans le fichier de a.num.
Quand on le trouve, on lit à partir de la ligne suivante. Chaque caractère est lu et il y a une incrémentation de \textit{x} et une décrémentation de \textit{y}.
Mais, à partir du moment où on passe à la ligne dans le fichier, on remet x à 0. 
Enfin, on retourne le plateau.


\section{Améliorations et recherches}
\begin{itemize}
\item Si on voulait utiliser un minimum de mémoire, il aurait fallu stocker l'historique sous forme de tableau d'éléments. 
En effet, ce qui change lorsque l'on joue au sokoban sont les caisses et le personnage. Donc, on aurait pu faire ceci : 
d'abord, \textbf{ACTIONJEU} possède un champs suplémentaire \textit{PLATEAU pMur} dans lequel il y a le plateau sans les caisses ni le personnage.
Ensuite, lors de l'initialisation de \textbf{ACTIONJEU}, on compte le nombre d'éléments \textit{nb} (personnage et caisses) ; on crée un tableau dynamique de la taille \textit{nb}.
Puis, on stocke les éléments. Lors des déplacements, on a juste à modifier le contenu des cases du tableau de l'historique.\newline

\item Pour être le plus réaliste possible, on pourrait créer une fonction supplémentaire qui teste si il y a des caisses ou des emplacements inacessibles pour le personnage. 
On pourrait la créer sur le même principe que int teste perso interieur entrepot. 
Et au lieu de tester si il y a des \emph{CASE\textunderscore TESTEE} sur les bords, il faudrait regarder chaque case du tableau. Si il existe une case où il y a caisse non testée alors on retourne 0 sinon 1.\newline

\item Lors de la création du jeu Sokoban, j'ai essayé de trouver les "failles" de mon programme. En effet, j'ai réfléchi au fait que l'utilisateur peut "nuire" au jeu : en essayant un fichier non valide ou avec des niveaux non valides par exemple. 
C'est pour cela qu'il faut prévoir toutes les actions possibles de l'utilisateur. \newline

\end{itemize}

\end{document}
