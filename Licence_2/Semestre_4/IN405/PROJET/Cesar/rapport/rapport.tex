% Clément CAUMES 21501810
% 30 avril 2017
% Projet Ave Cesar IN405
% Fichier de rapport du projet AVE CESAR

\documentclass[a4]{article}

\usepackage[utf8]{inputenc}
\usepackage[french]{babel}
\usepackage{amssymb}
\usepackage{caption}
\usepackage{amsmath}

\author{Clément Caumes 21501810}
\title{Rapport du projet Cesar}
\date{30 avril 2017}

\begin{document}
\maketitle

\section{Description des structures de données}
\subsection{Structure THREAD}
Cette structure correspond au type de l'argument dans la fonction de travail des employés. 
Elle contient le numéro du rang de la première lettre du mot à chiffrer/déchiffrer dans le buffer. 
Il y a également un pointeur sur ce buffer ainsi que sur le pas et le sens de chiffrement. 
On a aussi le thread associé à cet élément THREAD, avec des pointeurs sur un mutex et une condition afin de 
synchroniser les threads lors du chiffrement/déchiffrement du buffer. 

\subsection{Structure ANALYSE}
Cette structure contient les informations sur un message : le path du message, le pas et le sens de chiffrement, 
le pid du processus associé à ce message, le tube de communication (entre le chef d'équipe et le directeur). 
Il y a également le buffer qui sera alloué par le chef d'équipe ainsi que la liste de THREAD associés à chaque mot du buffer. 

\section{Implémentation générale}
Le projet est subdivisé en trois procédures principales qui correspondent au 
travail des trois différents acteurs : processus\_directeur, processus\_chef\_dequipe et threads\_employes.\newline

processus\_directeur est la procédure du directeur (processus du programme ./cesar) 
qui va lire le fichier principal dans lequel se trouvent les informations des messages 
à chiffrer/déchiffrer.\newline
Une liste est créée dans laquelle pour chaque élément il y a les informations pour chaque message. 
Le processus directeur va alors créer n processus fils pour les n messages. 
Ces processus fils vont faire la procédure de processus\_chef\_dequipe avec 
comme argument la structure ANALYSE contenant les informations du message attribué. \newline \newline

processus\_chef\_dequipe du message va alors stocker dans le buffer le message (lu grâce au path).
Ensuite, ce processus va compter le nombre de mots dans le buffer. Pour chaque mot, un thread va être créé 
dans lequel on envoie comme argument un THREAD contenant le pointeur du buffer, le numéro de la case du 
début du mot attribué, les pointeurs du pas et du sens de chiffrement, ainsi que des pointeurs sur le mutex 
et la condition du message (utiles pour la synchronisation des threads).\newline \newline

threads\_employes du mot va alors chiffrer/déchiffrer (selon le pas et le sens de 
chiffrement) chaque lettre du buffer à partir du rang auquel on l'a attribué, jusqu'à rencontrer un espace 
ou la fin de la chaîne de caractères de buffer. 
Tous les threads du message seront synchronisés (grâce à barriere). 
Lorsque chaque mot a été chiffré/déchiffré, on ferme les threads et on revient sur le processus\_chef\_dequipe du message. 
Si le message était un chiffrement alors le processus chef dequipe va créé un nouveau fichier-cypher dans lequel sera écrit 
le message-buffer chiffré. On utilise alors un tube anonyme qui va envoyer le buffer à son père (processus\_directeur). \newline
Quand on arrive dans le père (processus\_directeur), si le message était à déchiffrer, alors on lit dans le tube anonyme le buffer 
(envoyé par son fils) puis on l'affiche dans le terminal.

\section{Alternatives du projet}

Pour le projet, nous n'avons pas utilisé de mutex pour la sécurisation du buffer. En effet, en vue de l'implémentation du projet, 
les threads manipulent directement le buffer grâce au pointeur sur celui-ci et chaque thread manipule un intervalle particulier. Du coup, 
il n'y a pas vraiment de variable critique. 

\end{document}
